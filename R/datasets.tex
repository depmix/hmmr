 
\newpage
\section{Datasets}
%Alleen bedoeld voor onszelf, ideeen over data et cetera die we kunnen gebruiken

\begin{enumerate} 
	\item Introduction to mixture models
	\begin{enumerate}
		\item what are mixture models? 
		\item what is their relationship to HMMs?
		\item notation
	\end{enumerate}
	\item Definifing hidden Markov models
	\begin{enumerate}
		\item what are HMMs?
		\item what is hidden about HMMs?
		\item what is Markov about HMMs?
		\item Markov models
		\item hidden Markov model parameters
		\item important properties of hidden Markov models: 
		stationarity, homogeneity, transitions, observation
	\end{enumerate}
	\item Likelihood computation
	\item Estimating parameters with the EM algorithm
	\begin{enumerate}
		\item Formulating the estimation problem as a missing data 
		problem
		\item Details of the algorithm
		\item Defining and estimating models in R
	\end{enumerate}
	\item Constraining parameters using direct optimization of the 
	likelihood
	\begin{enumerate}
		\item direct optimization
		\item using gradients
		\item parameter starting values
		\item testing hypotheses by constraining parameters
	\end{enumerate}
	\item Posterior state sequences
	\begin{enumerate}
		\item Assigning data to sequences of states
		\item The Viterbi algorithm
	\end{enumerate}
	\item Inference
	\begin{enumerate}
		\item model selection
		\item parameter standard errors
	\end{enumerate}
\end{enumerate}

En dan hier onze eigen overwegingen, bovenstaand stuk is voor de
editor, dit voor onszelf.


Wat betreft de data sets is het misschien een idee om een aantal "key"
datasets te gebruiken die we hier een daar laten

Dat is zeker een goed idee; de speed-accuracy data die al bij depmix
zitten zijn uitermate geschikt voor een aantal dingen:

i) univariate gaussian modellen (alleen de RTs)

ii) mixed data (gaussian + binary)

iii) covariaten op de transitie kansen

iv) locale afhankelijkheid: in een 2-state model voor die data is er nog sprake van locale afhankelijkheid in de langzame component

v) ergodische modellen


Daarnaast heb ik discrimination learning data, goed voor de volgende
onderwerpen:

i) niet-ergodische modellen, transient and absorbing states

ii) binaire data

iii) mixtures van HMMs

Ik heb van Brenda Jansen nog een dataset met een microgenetic study
naar ontwikkeling op de balans taak: 10 herhaalde metingen over ca 10
weken met allerlei balans items en nog een of andere manipulatie (moet
ik nog bekijken, maar klinkt op papier als een hele mooie data set).

terugkomen.  Ik heb zelf wel wat leerdata die we kunnen gebruiken, met
name multiple cue probability learning data (weather prediction task,
stockmarket prediction task).  Misschien is het ook leuk iets met
"topic models" te doen.

Wat voor soort dingen kun je daar zoal mee illustreren?
Wpt data: univariaat, binary? Kun jij dit lijstje aanvullen?

Allebei zijn GLM's: de WPT voor logistische regressie en de SMPT voor
Gaussian/lineaire regressie.

Topic models zijn, voor zover ik weet, in esstentie multinomial
modellen.

Had Mark de Rooij trouwens niet iets interessants met voting data?

We kunnen ook in Wicken's boek kijken voor ideen.  Maar we moeten
misschien uitkijken niet alleen leerdata te gebruiken.  Klinische data
zou ook goed zijn.  Volgens mij heeft Peter Molenaar wel iets op dat
gebied, maar denk je dat hij het zou willen delen?

Ja, dat lijkt me geen probleem, ik weet niet zo goed wat hij heeft ...
Ik dacht wel ook aan de data van Gottmann over huwlijksdynamica, Han
heeft die data, maar daar zouden we zeker ook toestemming voor moeten
vragen.  Daarnaast heb ik altijd nog de Borkenau data waar ik wel eens
naar zou willen kijken.  Aan de andere kant: voor een boek zou ik
ernaar willen streven toch vooral gebruik te maken van reeds
gepubliceerd werk, anders moeten we nog echt nieuw werk gaan doen, dat
kan niet de bedoeling zijn (-; [Maar een beetje nieuw is wel leuk,
toch?]

De ESRC heeft ook publieke data waar we naar kunnen kijken.  Dan
moeten we wel toestemming vragen om het in een R package te zetten.

Dat klinkt ook interessant.  Wellicht voor een financiele of medische
data set?
 